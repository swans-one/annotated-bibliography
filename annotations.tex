\documentclass[12pt]{article}

\usepackage[style=authoryear]{biblatex}
\usepackage[dvipsnames,usenames]{color} % defined colors: 68 standard dvips colors
\usepackage{listings}

\addbibresource{bibliography.bib}

\newcommand{\note}[1] {
  \bigskip
  \noindent
  \emph{Note}: #1
  \bigskip
}

\newenvironment{annotatedcitation}[2]%
{\bigskip \subsubsection*{#1}  \fullcite{#2} \\ \smallskip \noindent}%
{\bigskip}

\newcommand{\pythonblock}{
  \lstset{
    basicstyle=\footnotesize,
    commentstyle=\color{BrickRed},
    keywordstyle=\color{Purple},
  }
}

\newcommand{\code}[1]{\texttt{#1}}

\begin{document}

\section{Papers}

  \begin{annotatedcitation}{Exploring the Link Among Behavior Interfention Plans, Treatment Integrity,
                            and Student Outcomes Under Natural Education Conditions}{cook12}

    I looked at this paper primarily as a study of the quality of
    their statistical methods and thinking. I did so as a favor for my
    mother.

    The paper dealt with the relationship between a mandated special
    education practice, Behavior Intervention Planning, and actual
    student outcomes in classrooms. It did so by recruiting pairs of
    participants, one of whom wrote the plan, and the other who
    implemented it with the special needs student.

    The plans were independently assessed by grad students for their
    objective quality. After implementing the plan, both participants
    rated the change in student outcomes and the extent to which the
    plan was followed. They found strong correlation between plan
    quality and student outcomes as well as implementation quality and
    student outcomes.

    Below is a brief list of the statistical methods they used in the
    paper.

    \begin{itemize}
      \item Descriptive Stats of participant demographics: Means,
        Standard Deviations, Frequencies
      \item Factor analysis for reducing the dimensionality of the
        measures of BIP quality.
      \item Correlational analysis of BIP plan, student behavior
        outcomes, student academic outcomes, and plan implementation
        integrity.
      \item Structural Equation Modeling (SEM) for assessing
        implementation of the plan as mediating the effect of plan
        quality on student outcomes.
    \end{itemize}

    I have problems with the correlational nature of this paper. While
    the correlations are interesting, they do not do enough to address
    the problem of bias in the researchers, as the participants were
    not blind to the treatment they were giving.

    Additionally, I find the use of SEM somewhat suspect, but mostly
    because I am not personally very familiar with its uses, strengths
    and weaknesses, not because it was necessarily a bad choice..

  \end{annotatedcitation}

  \begin{annotatedcitation}{Quasi-specific access of the potassium channel inactivation gate}{Venkataraman14}
    Voltage-activated Potassium Ion Channels---Shaker channels---cause
    fast inactivation of the cell.

    Their architecture is such that they have four:

    \begin{itemize}
      \item Inactivation gates
      \item Interacting residues
      \item Intracellular entryways
    \end{itemize}

    But inactivation only requires one binding.

    This paper shows that while the inactivation gate always threads
    through it's own intracellular entryway, it can bind with any of
    the four subunits.

    Additionally, by dissabling any one, two, three, or four of the
    sub-units, we can see that the inactivation response decreases
    uniformly---not with the specific sites dissabled, but rather
    decreases monotonically with the number of binding subunits
    disabled.

  \end{annotatedcitation}

\section{Books}

\subsection{Non-Fiction}

\subsection{Fiction}

  \begin{annotatedcitation}{Wool}{howey12}

    A dystopian novel. Centers on a number of characters living in a
    ``Silo.'' These characters do not realized that they are living in a
    post apocalyptic world until the protagonist, Juliette, is promotted
    to a leadership position within the community. She soon begins to
    realize that their silo is not alone.

    I rate Wool very positively.
  \end{annotatedcitation}

\subsection{Websites}

  \begin{annotatedcitation}{A Curious Course on Coroutines and Concurrency}{beazley09}

    Some key philosophical points:

    \begin{itemize}
      \item Generators are good for creating values, use them as part
        of a for loop.
      \item Coroutines are consumers of data, they are not related to
        iteration.
      \item Don't mix generators and coroutines. Coroutines are not
        related to iteration.
      \item Both can be used to set up pipes. Generators pull data
        through with iteration, coroutines push data through with
        send.
      \item Coroutines can be used to handle event driven problems. A
        stream of events acts as the 'driver' for a pipeline of
        processing code.
    \end{itemize}

    \note{much of the following is implemented more professionally in
    the \code{asyncio} package, included in the Python standard library
    as of version 3.4.}

    It's annoying to have to prime a coroutine with a call to
    \code{.send(None)}. To do this automatically, we can create a
    decorator that will prime our coroutines for us:

    \pythonblock
    \begin{lstlisting}[gobble=6, language=Python]
      def coroutine(func):
          def start(*args, **kwargs):
              cr = func(*args, **kwargs)
              cr.send(None)
              return cr
          return start
    \end{lstlisting}

    A useful pattern is to have a function that drives a coroutine,
    where the coroutine is passed in as a ``target'' possibly in a
    pipe, where one final coroutine acts as a ``sink,'' not passing values along.

    \pythonblock
    \begin{lstlisting}[gobble=6, language=Python]
      # The 'driver'
      def driver(file, target):
          for line in file:
              target.send(line)

      # the 'target'
      @coroutine
      def grep(pattern, target):
          while True:
              line = (yield)
              if pattern in line:
                  target.send(line)

      # the 'sink'
      @coroutine
      def lineprinter(target):
          while True:
              line = (yield)
              print line,
    \end{lstlisting}

    Coroutines can also be encapsulated into their own thread of
    execution. A python \code{Thread} takes a \code{target} as its
    argument and runs that target in a separate thread of
    execution. We use the native python threads to construct a
    ``\code{threaded}'' coroutine which transparently passes messages
    sent into the parent coroutine to a child ``\code{target}''
    coroutine, executing in a separate thread.

    \pythonblock
    \begin{lstlisting}[gobble=6, language=Python]
      from threading import Thread
      from Queue import Queue

      @coroutine
      def threaded(target):
          '''Pass signals sent to this coroutine to `target`,
          running in a new thread.
          '''
          # Create a message queue.
          messages = Queue()

          # Close over the message queue and the `target` argument
          # passed in to `threaded`
          def run_target():
              '''Pull messages from queue and pass them to target.
              Thread safe.
              '''
              while True:
                  item = messages.get()
                  if item is GeneratorExit:
                      target.close()
                      return
                  else:
                      target.send(item)

          # Create a new thread of execution, using the closure
          # defined above as the function to be run in the thread.
          Thread(target=run_target).start()

          # Act as a coroutine, pushing messages onto the queue,
          # shutting down child threads when shut down yourself.
          try:
              while True:
                  item = (yield)
                  messages.put(item)
          except GeneratorExit:
              messages.put(GeneratorExit)
    \end{lstlisting}

  \end{annotatedcitation}


\subsection{Video}

\end{document}
