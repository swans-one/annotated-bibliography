\documentclass[12pt]{article}

\usepackage[style=authoryear]{biblatex}
\usepackage[dvipsnames,usenames]{color} % defined colors: 68 standard dvips colors
\usepackage{listings}

\addbibresource{bibliography.bib}

\newcommand{\note}[1] {
  \bigskip
  \noindent
  \emph{Note}: #1
  \bigskip
}

\newenvironment{annotatedcitation}[2]%
{\bigskip \subsubsection*{#1}  \fullcite{#2} \\ \smallskip \noindent}%
{\bigskip}

\newcommand{\pythonblock}{
  \lstset{
    basicstyle=\footnotesize,
    commentstyle=\color{BrickRed},
    keywordstyle=\color{Purple},
  }
}

\newcommand{\code}[1]{\texttt{#1}}

\begin{document}

\section{Papers}

  \begin{annotatedcitation}{Quasi-specific access of the potassium channel inactivation gate}{Venkataraman14}
    Voltage-activated Potassium Ion Channels---Shaker channels---cause
    fast inactivation of the cell.

    Their architecture is such that they have four:

    \begin{itemize}
      \item Inactivation gates
      \item Interacting residues
      \item Intracellular entryways
    \end{itemize}

    But inactivation only requires one binding.

    This paper shows that while the inactivation gate always threads
    through it's own intracellular entryway, it can bind with any of
    the four subunits.

    Additionally, by dissabling any one, two, three, or four of the
    sub-units, we can see that the inactivation response decreases
    uniformly---not with the specific sites dissabled, but rather
    decreases monotonically with the number of binding subunits
    disabled.

  \end{annotatedcitation}

\section{Books}

\subsection{Non-Fiction}

\subsection{Fiction}

  \begin{annotatedcitation}{Wool}{howey12}

    A dystopian novel. Centers on a number of characters living in a
    ``Silo.'' These characters do not realized that they are living in a
    post apocalyptic world until the protagonist, Juliette, is promotted
    to a leadership position within the community. She soon begins to
    realize that their silo is not alone.

    I rate Wool very positively.
  \end{annotatedcitation}

\subsection{Websites}

  \begin{annotatedcitation}{A Curious Course on Coroutines and Concurrency}{beazley09}

    Some key philosophical points:

    \begin{itemize}
      \item Generators are good for creating values, use them as part
        of a for loop.
      \item Coroutines are consumers of data, they are not related to
        iteration.
      \item Don't mix generators and coroutines. Coroutines are not
        related to iteration.
      \item Both can be used to set up pipes. Generators pull data
        through with iteration, coroutines push data through with
        send.
      \item Coroutines can be used to handle event driven problems. A
        stream of events acts as the 'driver' for a pipeline of
        processing code.
    \end{itemize}

    \note{much of the following is implemented more professionally in
    the \code{asyncio} package, included in the Python standard library
    as of version 3.4.}

    It's annoying to have to prime a coroutine with a call to
    \code{.send(None)}. To do this automatically, we can create a
    decorator that will prime our coroutines for us:

    \pythonblock
    \begin{lstlisting}[gobble=6, language=Python]
      def coroutine(func):
          def start(*args, **kwargs):
              cr = func(*args, **kwargs)
              cr.send(None)
              return cr
          return start
    \end{lstlisting}

    A useful pattern is to have a function that drives a coroutine,
    where the coroutine is passed in as a ``target'' possibly in a
    pipe, where one final coroutine acts as a ``sink,'' not passing values along.

    \pythonblock
    \begin{lstlisting}[gobble=6, language=Python]
      # The 'driver'
      def driver(file, target):
          for line in file:
              target.send(line)

      # the 'target'
      @coroutine
      def grep(pattern, target):
          while True:
              line = (yield)
              if pattern in line:
                  target.send(line)

      # the 'sink'
      @coroutine
      def lineprinter(target):
          while True:
              line = (yield)
              print line,
    \end{lstlisting}

    Coroutines can also be encapsulated into their own thread of
    execution. A python \code{Thread} takes a \code{target} as its
    argument and runs that target in a separate thread of
    execution. We use the native python threads to construct a
    ``\code{threaded}'' coroutine which transparently passes messages
    sent into the parent coroutine to a child ``\code{target}''
    coroutine, executing in a separate thread.

    \pythonblock
    \begin{lstlisting}[gobble=6, language=Python]
      from threading import Thread
      from Queue import Queue

      @coroutine
      def threaded(target):
          '''Pass signals sent to this coroutine to `target`,
          running in a new thread.
          '''
          # Create a message queue.
          messages = Queue()

          # Close over the message queue and the `target` argument
          # passed in to `threaded`
          def run_target():
              '''Pull messages from queue and pass them to target.
              Thread safe.
              '''
              while True:
                  item = messages.get()
                  if item is GeneratorExit:
                      target.close()
                      return
                  else:
                      target.send(item)

          # Create a new thread of execution, using the closure
          # defined above as the function to be run in the thread.
          Thread(target=run_target).start()

          # Act as a coroutine, pushing messages onto the queue,
          # shutting down child threads when shut down yourself.
          try:
              while True:
                  item = (yield)
                  messages.put(item)
          except GeneratorExit:
              messages.put(GeneratorExit)
    \end{lstlisting}

  \end{annotatedcitation}


\subsection{Video}

\end{document}
